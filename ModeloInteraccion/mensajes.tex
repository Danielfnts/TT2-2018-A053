%\DONE{} 
	En esta sección se describen los mensajes utilizados en el prototipo actual del sistema. Los mensajes se refieren a todos
	aquellos avisos que el sistema muestra al actor a través de la pantalla debido a diversas
	razones, por ejemplo: informar acerca de algún fallo en el sistema o para notificar acerca de alguna operación importante sobre
	la información.

%===========================================================removiendo los puntos generados
\subsection{Parámetros comunes}
    Cuando un mensaje es recurrente se parametrizan sus elementos, por ejemplo los mensajes: ``Aún no existen registros de {\em escuelas} en el sistema.'', ``Aún no existen registros de {\em responsables del programa} en el sistema.'', 
    ``Aún no existen registros de {\em integrantes de líneas de acción} en el sistema.'', tienen una estructura similar 
    por lo que para definir el mensaje se utilizan parámentros, con el objetivo de que el mensaje sea genérico y  
    pueda utilizarse en todos los casos que se considere necesario.\\
    
    Los parámetros también se utilizan cuando la redacción del mensaje tiene datos que son ingresados por el actor o que dependen del resultado de la operación, por ejemplo: 
    ``La {\em escuela  15DPR2497K} ha sido {\em modificada} exitosamente.''. En este caso la redacción se presenta parametrizada de la forma: ``DETERMINADO ENTIDAD VALOR ha sido OPERACIÓN exitosamente.'' y los 
    parámetros se describen de la siguiente forma:
    
    \begin{itemize}
	\item DETERMINADO ENTIDAD: Es un artículo determinado más el nombre de la entidad sobre la cual se realizó la acción.
	\item VALOR: Es el valor asignado al atributo de la entidad, generalmente es el nombre o la clave.
	\item OPERACIÓN: Es la acción que el actor solicitó realizar.
    \end{itemize}

    En el ejemplo anterior se hace referencia a VALOR, es decir: {\em 15DPR2497K} es el {\bf valor}  de la entidad {\bf escuela}. Cada mensaje enlista los parámetros 
    que utiliza, sin embargo aquí se definen los más comunes a fin de simplificar la descripción de los mensajes:

    \begin{description}
	\item [ARTÍCULO:] Se refiere a un {\em artículo} el cual puede ser DETERMINADO (El $\mid$ La $\mid$ Lo $\mid$ Los $\mid$ Las) o INDETERMINADO (Un $\mid$ Una $\mid$ 
	Uno $\mid$ Unos $\mid$Unas) se aplica generalmente sobre una ENTIDAD, ATRIBUTO o VALOR.
	\item [CAMPO:] Se refiere a un campo del formulario. Por lo regular es el nombre de un atributo en una entidad.
	\item [CONDICIÓN:] Define una expresión booleana cuyo resultado deriva en {\em falso} o {\em verdadero} y suele ser la causa del mensaje.
	\item [DATO:] Es un sustantivo y generalmente se refiere a un atributo de una entidad descrito en el modelo estructural del negocio, por ejemplo: número de incendio,
	brigada de apoyo del incendio, uso de suelo autorizado del predio, etc. %ATRIBUTO
	\item [ENTIDAD:] Es un sustantivo y generalmente se refiere a una entidad del modelo estructural del negocio, por ejemplo: incendio, pago por servicios ambientales hidrológicos, reforestación, etc.
	\item [OPERACIÓN:] Se refiere a una acción que se debe realizar sobre los datos de una o varias entidades. Por ejemplo: registrar, eliminar, actualizar, etc. Comúnmente 
	la OPERACIÓN va concatenada con el sustantivo, por ejemplo: Registro de un nuevo beneficio, registro de una actividad, eliminar una tarea, etc.
	\item [VALOR:] Es un sustantivo concreto y generalmente se refiere a un valor en específico. Por ejemplo: ``2014-003'', que es un valor concreto del DATO de la 
	ENTIDAD ``incendio''.
	\item [TAMAÑO:] Es el tamaño del atributo de una entidad, el cual se encuentra definido en el diccionario de datos.
	\item [MOTIVO:] Es una explicación acerca de la operación que se pretende realizar.
    \end{description}


\subsection{Mensajes a través de la pantalla}

%===========================  MSG1 ==================================
\begin{mensaje}{MSG1}{Operación Exitosa}{Confirmación}
%    \item[Ubicación:] Formulario. Se muestra en la parte superior del formulario.
    \item[Estatus:] Terminado
    \item[Objetivo:] Notificar al actor que la acción solicitada fue realizada exitosamente.
    \item[Redacción:] DETERMINADO ENTIDAD VALOR ha sido OPERACIÓN exitosamente.
    \item[Parámetros:] El mensaje se muestra con base en los siguientes parámetros:
    \begin{Citemize}
	\item DETERMINADO ENTIDAD: Es un artículo determinado más el nombre de la entidad sobre la cual se realizó la acción.
	\item VALOR: Es el valor asignado al atributo de la entidad, generalmente es el nombre o la clave.
	\item OPERACIÓN: Es la acción que el actor solicitó realizar.
    \end{Citemize}
    \item[Ejemplo:] {\em La escuela 15DPR2497K} ha sido {\em registrada} exitosamente.
\end{mensaje}

%===========================  MSG2 ==================================
\begin{mensaje}{MSG2}{Falta un dato requerido para efectuar la operación solicitada}{Error}
	%    \item[Ubicación:] Formulario. Se muestra en la parte superior del formulario.
	\item[Estatus:] Terminado
	\item[Objetivo:] Notificar al actor que no falta un dato marcado como obligatorio.
	\item[Redacción:] Campo obligatorio.
%	\item[Parámetros:] El mensaje se muestra con base en los siguientes parámetros:
%	\begin{Citemize}
%		\item DETERMINADO ENTIDAD: Es un artículo determinado más el nombre de la entidad sobre la cual se realizó la acción.
%		\item VALOR: Es el valor asignado al atributo de la entidad, generalmente es el nombre o la clave.
%		\item OPERACIÓN: Es la acción que el actor solicitó realizar.
%	\end{Citemize}
%	\item[Ejemplo:] {\em La escuela 15DPR2497K} ha sido {\em registrada} exitosamente.
\end{mensaje}

%===========================  MSG3 ==================================
\begin{mensaje}{MSG3}{Nombre de usuario y/o contraseña incorrecto}{Error}
	%    \item[Ubicación:] Formulario. Se muestra en la parte superior del formulario.
	\item[Estatus:] Terminado
	\item[Objetivo:] Notificar al actor que ingresó un nombre de usuario y/o contraseña incorrecta
	\item[Redacción:] DETERMINADO ENTIDAD VALOR está incorrecto.
	\item[Parámetros:] El mensaje se muestra con base en los siguientes parámetros:
	\begin{Citemize}
		\item DETERMINADO ENTIDAD: Es un artículo determinado más el nombre de la entidad sobre la cual se realizó la acción.
%		\item VALOR: Es el valor asignado al atributo de la entidad, generalmente es el nombre o la clave.
%		\item OPERACIÓN: Es la acción que el actor solicitó realizar.
	\end{Citemize}
	\item[Ejemplo:] La contraseña ingresada está incorrecta.
\end{mensaje}

%===========================  MSG4 ==================================
\begin{mensaje}{MSG4}{No existe correo}{Error}
	%    \item[Ubicación:] Formulario. Se muestra en la parte superior del formulario.
	\item[Estatus:] Terminado
	\item[Objetivo:] Notificar al actor que el correo ingresado no existe.
	\item[Redacción:] DETERMINADO ENTIDAD VALOR no existe en el sistema.
	\item[Parámetros:] El mensaje se muestra con base en los siguientes parámetros:
	\begin{Citemize}
		\item DETERMINADO ENTIDAD: Es un artículo determinado más el nombre de la entidad sobre la cual se realizó la acción.
%		\item VALOR: Es el valor asignado al atributo de la entidad, generalmente es el nombre o la clave.
%		\item OPERACIÓN: Es la acción que el actor solicitó realizar.
	\end{Citemize}
	\item[Ejemplo:] El correo que ingresó no existe en el sistema.
\end{mensaje}

%===========================  MSG5 ==================================
\begin{mensaje}{MSG5}{Obtener datos}{Confirmación}
	%    \item[Ubicación:] Formulario. Se muestra en la parte superior del formulario.
	\item[Estatus:] Terminado
	\item[Objetivo:] Notificar al actor que el sistema operativo desea obtener su número celular
	\item[Redacción:] El sistema operativo desea obtener tu número.
%	\item[Parámetros:] El mensaje se muestra con base en los siguientes parámetros:
%	\begin{Citemize}
%		\item DETERMINADO ENTIDAD: Es un artículo determinado más el nombre de la entidad sobre la cual se realizó la acción.
%		\item VALOR: Es el valor asignado al atributo de la entidad, generalmente es el nombre o la clave.
%		\item OPERACIÓN: Es la acción que el actor solicitó realizar.
%	\end{Citemize}
%	\item[Ejemplo:] {\em La escuela 15DPR2497K} ha sido {\em registrada} exitosamente.
\end{mensaje}

%===========================  MSG6 ==================================
\begin{mensaje}{MSG6}{Faltan datos en la contraseña}{Error}
	%    \item[Ubicación:] Formulario. Se muestra en la parte superior del formulario.
	\item[Estatus:] Terminado
	\item[Objetivo:] Notificar al actor que la contraseña ingresada no cumple con el formato correcto.
	\item[Redacción:] La contraseña debe tener al menos 6 caracteres, una mayúscula, una minúscula, un número y un caracter especial.
%	\item[Parámetros:] El mensaje se muestra con base en los siguientes parámetros:
%	\begin{Citemize}
%		\item DETERMINADO ENTIDAD: Es un artículo determinado más el nombre de la entidad sobre la cual se realizó la acción.
%		\item VALOR: Es el valor asignado al atributo de la entidad, generalmente es el nombre o la clave.
%		\item OPERACIÓN: Es la acción que el actor solicitó realizar.
%	\end{Citemize}
%	\item[Ejemplo:] La contras.
\end{mensaje}

%===========================  MSG7 ==================================
\begin{mensaje}{MSG7}{Obtener foto}{Confirmación}
	%    \item[Ubicación:] Formulario. Se muestra en la parte superior del formulario.
	\item[Estatus:] Terminado
	\item[Objetivo:] Preguntar al actor si desea obtener otra foto.
	\item[Redacción:] ¿Desea obtener otra foto?
%	\item[Parámetros:] El mensaje se muestra con base en los siguientes parámetros:
%	\begin{Citemize}
%		\item DETERMINADO ENTIDAD: Es un artículo determinado más el nombre de la entidad sobre la cual se realizó la acción.
%		\item VALOR: Es el valor asignado al atributo de la entidad, generalmente es el nombre o la clave.
%		\item OPERACIÓN: Es la acción que el actor solicitó realizar.
%	\end{Citemize}
%	\item[Ejemplo:] {\em La escuela 15DPR2497K} ha sido {\em registrada} exitosamente.
\end{mensaje}

%===========================  MSG8 ==================================
\begin{mensaje}{MSG8}{No se pudieron obtener los datos}{Error}
	%    \item[Ubicación:] Formulario. Se muestra en la parte superior del formulario.
	\item[Estatus:] Terminado
	\item[Objetivo:] Notificar al actor que el sistema no pudo obtener los datos requeridos.
	\item[Redacción:] El sistema o pudo obtener los datos requeridos.
%	\item[Parámetros:] El mensaje se muestra con base en los siguientes parámetros:
%	\begin{Citemize}
%		\item DETERMINADO ENTIDAD: Es un artículo determinado más el nombre de la entidad sobre la cual se realizó la acción.
%		\item VALOR: Es el valor asignado al atributo de la entidad, generalmente es el nombre o la clave.
%		\item OPERACIÓN: Es la acción que el actor solicitó realizar.
%	\end{Citemize}
%	\item[Ejemplo:] {\em La escuela 15DPR2497K} ha sido {\em registrada} exitosamente.
\end{mensaje}

%===========================  MSG9 ==================================
\begin{mensaje}{MSG9}{No tiene tratamiento}{Error}
	%    \item[Ubicación:] Formulario. Se muestra en la parte superior del formulario.
	\item[Estatus:] Terminado
	\item[Objetivo:] Notificar al actor que no tiene tratamientos registrados
	\item[Redacción:] No cuanta aún con un tratamiento registrado.
%	\item[Parámetros:] El mensaje se muestra con base en los siguientes parámetros:
%	\begin{Citemize}
%		\item DETERMINADO ENTIDAD: Es un artículo determinado más el nombre de la entidad sobre la cual se realizó la acción.
%		\item VALOR: Es el valor asignado al atributo de la entidad, generalmente es el nombre o la clave.
%		\item OPERACIÓN: Es la acción que el actor solicitó realizar.
%	\end{Citemize}
%	\item[Ejemplo:] {\em La escuela 15DPR2497K} ha sido {\em registrada} exitosamente.
\end{mensaje}

%===========================  MSG10 ==================================
\begin{mensaje}{MSG10}{Correo electrónico nuevo}{Confirmación}
	%    \item[Ubicación:] Formulario. Se muestra en la parte superior del formulario.
	\item[Estatus:] Terminado
	\item[Objetivo:] Notificar al actor que el correo electrónico que desea ingresar es nuevo.
	\item[Redacción:] El correo electrónico que quiere ingresar es nuevo.
%	\item[Parámetros:] El mensaje se muestra con base en los siguientes parámetros:
%	\begin{Citemize}
%		\item DETERMINADO ENTIDAD: Es un artículo determinado más el nombre de la entidad sobre la cual se realizó la acción.
%		\item VALOR: Es el valor asignado al atributo de la entidad, generalmente es el nombre o la clave.
%		\item OPERACIÓN: Es la acción que el actor solicitó realizar.
%	\end{Citemize}
%	\item[Ejemplo:] {\em La escuela 15DPR2497K} ha sido {\em registrada} exitosamente.
\end{mensaje}

%===========================  MSG11 ==================================
\begin{mensaje}{MSG11}{No hay rol}{Error}
	%    \item[Ubicación:] Formulario. Se muestra en la parte superior del formulario.
	\item[Estatus:] Terminado
	\item[Objetivo:] Notificar al actor que no seleccionó ningún rol.
	\item[Redacción:] No ha seleccionado ningún rol.
%	\item[Parámetros:] El mensaje se muestra con base en los siguientes parámetros:
%	\begin{Citemize}
%		\item DETERMINADO ENTIDAD: Es un artículo determinado más el nombre de la entidad sobre la cual se realizó la acción.
%		\item VALOR: Es el valor asignado al atributo de la entidad, generalmente es el nombre o la clave.
%		\item OPERACIÓN: Es la acción que el actor solicitó realizar.
%	\end{Citemize}
%	\item[Ejemplo:] {\em La escuela 15DPR2497K} ha sido {\em registrada} exitosamente.
\end{mensaje}

%===========================  MSG12 ==================================
\begin{mensaje}{MSG12}{Cédula Profesional no válida}{Error}
	%    \item[Ubicación:] Formulario. Se muestra en la parte superior del formulario.
	\item[Estatus:] Terminado
	\item[Objetivo:] Notificar al actor que la cédula profesional que ingresó no es válida.
	\item[Redacción:] La cédula profesional que intenta ingresar no es válida.
%	\item[Parámetros:] El mensaje se muestra con base en los siguientes parámetros:
%	\begin{Citemize}
%		\item DETERMINADO ENTIDAD: Es un artículo determinado más el nombre de la entidad sobre la cual se realizó la acción.
%		\item VALOR: Es el valor asignado al atributo de la entidad, generalmente es el nombre o la clave.
%		\item OPERACIÓN: Es la acción que el actor solicitó realizar.
%	\end{Citemize}
%	\item[Ejemplo:] {\em La escuela 15DPR2497K} ha sido {\em registrada} exitosamente.
\end{mensaje}

%===========================  MSG13 ==================================
\begin{mensaje}{MSG13}{No existe el doctor}{Error}
	%    \item[Ubicación:] Formulario. Se muestra en la parte superior del formulario.
	\item[Estatus:] Terminado
	\item[Objetivo:] Notificar al actor que el doctor que quiere ingresar no existe.
	\item[Redacción:] El doctor que ingresó no existe, verifique el número de cédula.
%	\item[Parámetros:] El mensaje se muestra con base en los siguientes parámetros:
%	\begin{Citemize}
%		\item DETERMINADO ENTIDAD: Es un artículo determinado más el nombre de la entidad sobre la cual se realizó la acción.
%		\item VALOR: Es el valor asignado al atributo de la entidad, generalmente es el nombre o la clave.
%		\item OPERACIÓN: Es la acción que el actor solicitó realizar.
%	\end{Citemize}
%	\item[Ejemplo:] {\em La escuela 15DPR2497K} ha sido {\em registrada} exitosamente.
\end{mensaje}

%===========================  MSG14 ==================================
\begin{mensaje}{MSG14}{Medicamento agregado pero incompleto}{Confirmación}
	%    \item[Ubicación:] Formulario. Se muestra en la parte superior del formulario.
	\item[Estatus:] Terminado
	\item[Objetivo:] Notificar al actor que falta por completar el registro de un medicamento.
	\item[Redacción:] No ha completado el registro de su medicamento.
%	\item[Parámetros:] El mensaje se muestra con base en los siguientes parámetros:
%	\begin{Citemize}
%		\item DETERMINADO ENTIDAD: Es un artículo determinado más el nombre de la entidad sobre la cual se realizó la acción.
%		\item VALOR: Es el valor asignado al atributo de la entidad, generalmente es el nombre o la clave.
%		\item OPERACIÓN: Es la acción que el actor solicitó realizar.
%	\end{Citemize}
%	\item[Ejemplo:] {\em La escuela 15DPR2497K} ha sido {\em registrada} exitosamente.
\end{mensaje}

%===========================  MSG15 ==================================
\begin{mensaje}{MSG15}{Doctor existente}{Error}
	%    \item[Ubicación:] Formulario. Se muestra en la parte superior del formulario.
	\item[Estatus:] Terminado
	\item[Objetivo:] Notificar al actor que el doctor que desea registrar ya existe.
	\item[Redacción:] El doctor que desea registrar ya existe.
%	\item[Parámetros:] El mensaje se muestra con base en los siguientes parámetros:
%	\begin{Citemize}
%		\item DETERMINADO ENTIDAD: Es un artículo determinado más el nombre de la entidad sobre la cual se realizó la acción.
%		\item VALOR: Es el valor asignado al atributo de la entidad, generalmente es el nombre o la clave.
%		\item OPERACIÓN: Es la acción que el actor solicitó realizar.
%	\end{Citemize}
%	\item[Ejemplo:] {\em La escuela 15DPR2497K} ha sido {\em registrada} exitosamente.
\end{mensaje}

%===========================  MSG16 ==================================
\begin{mensaje}{MSG16}{Eliminar doctor}{Confirmación}
	%    \item[Ubicación:] Formulario. Se muestra en la parte superior del formulario.
	\item[Estatus:] Terminado
	\item[Objetivo:] Solicitar la confirmación del actor para eliminar un doctor.
	\item[Redacción:] ¿Está seguro que desea eliminar el doctor seleccionado?
%	\item[Parámetros:] El mensaje se muestra con base en los siguientes parámetros:
%	\begin{Citemize}
%		\item DETERMINADO ENTIDAD: Es un artículo determinado más el nombre de la entidad sobre la cual se realizó la acción.
%		\item VALOR: Es el valor asignado al atributo de la entidad, generalmente es el nombre o la clave.
%		\item OPERACIÓN: Es la acción que el actor solicitó realizar.
%	\end{Citemize}
%	\item[Ejemplo:] {\em La escuela 15DPR2497K} ha sido {\em registrada} exitosamente.
\end{mensaje}

%===========================  MSG17 ==================================
\begin{mensaje}{MSG17}{Tratamiento pasaran a estado incompleto}{Confirmación}
	%    \item[Ubicación:] Formulario. Se muestra en la parte superior del formulario.
	\item[Estatus:] Terminado
	\item[Objetivo:] Notificar al actor que el tratamiento pasará a estado incompleto.
	\item[Redacción:] El tratamiento pasará a estado incompleto.
%	\item[Parámetros:] El mensaje se muestra con base en los siguientes parámetros:
%	\begin{Citemize}
%		\item DETERMINADO ENTIDAD: Es un artículo determinado más el nombre de la entidad sobre la cual se realizó la acción.
%		\item VALOR: Es el valor asignado al atributo de la entidad, generalmente es el nombre o la clave.
%		\item OPERACIÓN: Es la acción que el actor solicitó realizar.
%	\end{Citemize}
%	\item[Ejemplo:] {\em La escuela 15DPR2497K} ha sido {\em registrada} exitosamente.
\end{mensaje}

%===========================  MSG18 ==================================
\begin{mensaje}{MSG18}{No hay doctores}{Error}
	%    \item[Ubicación:] Formulario. Se muestra en la parte superior del formulario.
	\item[Estatus:] Terminado
	\item[Objetivo:] Notificar al actor que no cuenta con doctores asociados a su cuenta.
	\item[Redacción:] No tiene doctores asociados a su cuenta.
%	\item[Parámetros:] El mensaje se muestra con base en los siguientes parámetros:
%	\begin{Citemize}
%		\item DETERMINADO ENTIDAD: Es un artículo determinado más el nombre de la entidad sobre la cual se realizó la acción.
%		\item VALOR: Es el valor asignado al atributo de la entidad, generalmente es el nombre o la clave.
%		\item OPERACIÓN: Es la acción que el actor solicitó realizar.
%	\end{Citemize}
%	\item[Ejemplo:] {\em La escuela 15DPR2497K} ha sido {\em registrada} exitosamente.
\end{mensaje}

%===========================  MSG19 ==================================
\begin{mensaje}{MSG19}{Toma realizada}{Confirmación}
	%    \item[Ubicación:] Formulario. Se muestra en la parte superior del formulario.
	\item[Estatus:] Terminado
	\item[Objetivo:] Notificar al actor que realizó la toma de su medicamento.
	\item[Redacción:] Ha realizado la toma del medicamento exitosamente.
%	\item[Parámetros:] El mensaje se muestra con base en los siguientes parámetros:
%	\begin{Citemize}
%		\item DETERMINADO ENTIDAD: Es un artículo determinado más el nombre de la entidad sobre la cual se realizó la acción.
%		\item VALOR: Es el valor asignado al atributo de la entidad, generalmente es el nombre o la clave.
%		\item OPERACIÓN: Es la acción que el actor solicitó realizar.
%	\end{Citemize}
%	\item[Ejemplo:] {\em La escuela 15DPR2497K} ha sido {\em registrada} exitosamente.
\end{mensaje}

%===========================  MSG20 ==================================
\begin{mensaje}{MSG20}{Aviso a contacto de emergencia}{Confirmación}
	%    \item[Ubicación:] Formulario. Se muestra en la parte superior del formulario.
	\item[Estatus:] Terminado
	\item[Objetivo:] Alertar a un contacto de emergencia que su paciente no ha consumido un medicamento de alta importancia.
	\item[Redacción:] El paciente no ha consumido correctamente sus medicamentos.
%	\item[Parámetros:] El mensaje se muestra con base en los siguientes parámetros:
%	\begin{Citemize}
%		\item DETERMINADO ENTIDAD: Es un artículo determinado más el nombre de la entidad sobre la cual se realizó la acción.
%		\item VALOR: Es el valor asignado al atributo de la entidad, generalmente es el nombre o la clave.
%		\item OPERACIÓN: Es la acción que el actor solicitó realizar.
%	\end{Citemize}
%	\item[Ejemplo:] {\em La escuela 15DPR2497K} ha sido {\em registrada} exitosamente.
\end{mensaje}

%===========================  MSG21 ==================================
\begin{mensaje}{MSG21}{No hay auxiliares}{Error}
	%    \item[Ubicación:] Formulario. Se muestra en la parte superior del formulario.
	\item[Estatus:] Terminado
	\item[Objetivo:] Notificar al actor que no tiene auxiliares asociados a su cuenta.
	\item[Redacción:] No tiene auxiliares asociados a su cuenta.
%	\item[Parámetros:] El mensaje se muestra con base en los siguientes parámetros:
%	\begin{Citemize}
%		\item DETERMINADO ENTIDAD: Es un artículo determinado más el nombre de la entidad sobre la cual se realizó la acción.
%		\item VALOR: Es el valor asignado al atributo de la entidad, generalmente es el nombre o la clave.
%		\item OPERACIÓN: Es la acción que el actor solicitó realizar.
%	\end{Citemize}
%	\item[Ejemplo:] {\em La escuela 15DPR2497K} ha sido {\em registrada} exitosamente.
\end{mensaje}

%===========================  MSG22 ==================================
\begin{mensaje}{MSG22}{Eliminar elemento}{Confirmación}
	%    \item[Ubicación:] Formulario. Se muestra en la parte superior del formulario.
	\item[Estatus:] Terminado
	\item[Objetivo:] Solicitar la confirmación del actor para eliminar un elemento.
	\item[Redacción:] ¿Está seguro que desea eliminar este elemento?.
%	\item[Parámetros:] El mensaje se muestra con base en los siguientes parámetros:
%	\begin{Citemize}
%		\item DETERMINADO ENTIDAD: Es un artículo determinado más el nombre de la entidad sobre la cual se realizó la acción.
%		\item VALOR: Es el valor asignado al atributo de la entidad, generalmente es el nombre o la clave.
%		\item OPERACIÓN: Es la acción que el actor solicitó realizar.
%	\end{Citemize}
%	\item[Ejemplo:] {\em La escuela 15DPR2497K} ha sido {\em registrada} exitosamente.
\end{mensaje}

%===========================  MSG23 ==================================
\begin{mensaje}{MSG23}{No hay contactos de emergencia}{Error}
	%    \item[Ubicación:] Formulario. Se muestra en la parte superior del formulario.
	\item[Estatus:] Terminado
	\item[Objetivo:] Notificar al actor que no hay contactos de emergencia asociados a su cuenta.
	\item[Redacción:] No tiene contactos de emergencia asociados a su cuenta.
%	\item[Parámetros:] El mensaje se muestra con base en los siguientes parámetros:
%	\begin{Citemize}
%		\item DETERMINADO ENTIDAD: Es un artículo determinado más el nombre de la entidad sobre la cual se realizó la acción.
%		\item VALOR: Es el valor asignado al atributo de la entidad, generalmente es el nombre o la clave.
%		\item OPERACIÓN: Es la acción que el actor solicitó realizar.
%	\end{Citemize}
%	\item[Ejemplo:] {\em La escuela 15DPR2497K} ha sido {\em registrada} exitosamente.
\end{mensaje}

%===========================  MSG24 ==================================
\begin{mensaje}{MSG24}{Validación de contraseña incorrecta}{Error}
	%    \item[Ubicación:] Formulario. Se muestra en la parte superior del formulario.
	\item[Estatus:] Terminado
	\item[Objetivo:] Notificar al actor que la validación de la contraseña no coincide con la contraseña ingresada.
	\item[Redacción:] La validación de la contraseña no coincide con la contraseña ingresada.
%	\item[Parámetros:] El mensaje se muestra con base en los siguientes parámetros:
%	\begin{Citemize}
%		\item DETERMINADO ENTIDAD: Es un artículo determinado más el nombre de la entidad sobre la cual se realizó la acción.
%		\item VALOR: Es el valor asignado al atributo de la entidad, generalmente es el nombre o la clave.
%		\item OPERACIÓN: Es la acción que el actor solicitó realizar.
%	\end{Citemize}
%	\item[Ejemplo:] {\em La escuela 15DPR2497K} ha sido {\em registrada} exitosamente.
\end{mensaje}

%===========================  MSG25==================================
\begin{mensaje}{MSG25}{No hay recordatorios}{Error}
	%    \item[Ubicación:] Formulario. Se muestra en la parte superior del formulario.
	\item[Estatus:] Terminado
	\item[Objetivo:] Notificar al actor que no ha configurado ningún recordatorio
	\item[Redacción:] No tiene recordatorios configurados.
%	\item[Parámetros:] El mensaje se muestra con base en los siguientes parámetros:
%	\begin{Citemize}
%		\item DETERMINADO ENTIDAD: Es un artículo determinado más el nombre de la entidad sobre la cual se realizó la acción.
%		\item VALOR: Es el valor asignado al atributo de la entidad, generalmente es el nombre o la clave.
%		\item OPERACIÓN: Es la acción que el actor solicitó realizar.
%	\end{Citemize}
%	\item[Ejemplo:] {\em La escuela 15DPR2497K} ha sido {\em registrada} exitosamente.
\end{mensaje}

%===========================  MSG26 ==================================
\begin{mensaje}{MSG26}{No hay alertas}{Error}
	%    \item[Ubicación:] Formulario. Se muestra en la parte superior del formulario.
	\item[Estatus:] Terminado
	\item[Objetivo:] Notificar al actor que no ha configurado ninguna alerta.
	\item[Redacción:] No tiene alertas configuradas.
%	\item[Parámetros:] El mensaje se muestra con base en los siguientes parámetros:
%	\begin{Citemize}
%		\item DETERMINADO ENTIDAD: Es un artículo determinado más el nombre de la entidad sobre la cual se realizó la acción.
%		\item VALOR: Es el valor asignado al atributo de la entidad, generalmente es el nombre o la clave.
%		\item OPERACIÓN: Es la acción que el actor solicitó realizar.
%	\end{Citemize}
%	\item[Ejemplo:] {\em La escuela 15DPR2497K} ha sido {\em registrada} exitosamente.
\end{mensaje}
