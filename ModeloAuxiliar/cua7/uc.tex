\begin{UseCase}{CUA6}{Consultar Tratamiento del Paciente}
    {
    	
    	Un Auxiliar tendrá una lista con los tratamientos recetados a sus pacientes que auxilian, tanto los que están en estado activo, incompleto, interrumpido y terminado. De esta manera sabrá cuantos y cuáles son sus tratamientos, su estado y la información de estos, como nombre de tratamiento, paciente al que fue asignado, fecha de expedición y número de medicamentos
    	
    	
    }
    \UCitem{Versión}{1.0}
    \UCccsection{Auxiliar}
    \UCitem{Autor}{Daniel Josue Fuentes Hernández}
    \UCccitem{Evaluador}{Victor Arquimedes Estrada Machuca}
    \UCitem{Operación}{Consulta de Tratamiento}
    \UCccitem{Prioridad}{Alta}
    \UCccitem{Complejidad}{Baja}
    \UCccitem{Volatilidad}{Baja}
    \UCccitem{Madurez}{Alta}
    \UCitem{Estatus}{Terminado}
    \UCitem{Fecha del último estatus}{09 de Abril del 2019}

%% Copie y pegue este bloque tantas veces como revisiones tenga el caso de uso.
%% Esta sección la debe llenar solo el Revisor
% %--------------------------------------------------------
% 	\UCccsection{Revisión Versión XX} % Anote la versión que se revisó.
% 	% FECHA: Anote la fecha en que se terminó la revisión.
% 	\UCccitem{Fecha}{Fecha en que se termino la revisión} 
% 	% EVALUADOR: Coloque el nombre completo de quien realizó la revisión.
% 	\UCccitem{Evaluador}{Nombre de quien revisó}
% 	% RESULTADO: Coloque la palabra que mas se apegue al tipo de acción que el analista debe realizar.
% 	\UCccitem{Resultado}{Corregir, Desechar, Rehacer todo, terminar.}
% 	% OBSERVACIONES: Liste los cambios que debe realizar el Analista.
% 	\UCccitem{Observaciones}{
% 		\begin{UClist}
% 			% PC: Petición de Cambio, describa el trabajo a realizar, si es posible indique la causa de la PC. Opcionalmente especifique la fecha en que considera razonable que se deba terminar la PC. No olvide que la numeración no se debe reiniciar en una segunda o tercera revisión.
% 			\RCitem{PC1}{\TODO{Descripción del pendiente}}{Fecha de entrega}
% 			\RCitem{PC2}{\TODO{Descripción del pendiente}}{Fecha de entrega}
% 			\RCitem{PC3}{\TODO{Descripción del pendiente}}{Fecha de entrega}
% 		\end{UClist}		
% 	}
% %--------------------------------------------------------
% %--------------------------------------------------------
% 	\UCccsection{Revisión Versión 0.3} % Anote la versión que se revisó.
% 	% FECHA: Anote la fecha en que se terminó la revisión.
% 	\UCccitem{Fecha}{04/11/2014} 
% 	% EVALUADOR: Coloque el nombre completo de quien realizó la revisión.
% 	\UCccitem{Evaluador}{Natalia Giselle Hernández Sánchez}
% 	% RESULTADO: Coloque la palabra que mas se apegue al tipo de acción que el analista debe realizar.
% 	\UCccitem{Resultado}{Corregir}
% 	% OBSERVACIONES: Liste los cambios que debe realizar el Analista.
% 	\UCccitem{Observaciones}{
% 		\begin{UClist}
% 			% PC: Petición de Cambio, describa el trabajo a realizar, si es posible indique la causa de la PC. Opcionalmente especifique la fecha en que considera razonable que se deba terminar la PC. No olvide que la numeración no se debe reiniciar en una segunda o tercera revisión.
% 			\RCitem{PC2}{\DONE{Cambiar en la sección de entradas el nombre del atributo ``Nombre de usuario'' y la redirección}}{04/11}
% 			\RCitem{PC3}{\DONE{La liga de ``Contraseña'' está rota}}{04/11}
% 			\RCitem{PC4}{\DONE{Hay oraciones que no tienen punto al final, como las precondiciones}}{04/11}
% 			\RCitem{PC5}{\DONE{Quitar la palabra ``interna'' de las viñetasde los menús en la sección de postcondiciones}}{04/11}
% 			\RCitem{PC6}{\DONE{En las postcondiciones dice ``perfl''}}{04/11}
% 			\RCitem{PC7}{\DONE{Falta punto en el primer paso de la TP}}{05/11}
% 			\RCitem{PC8}{\DONE{El paso 6 de la trayectoria principal no debería de estar debido a que el nombre de usuario no siempre es la cct}}{05/11}
% 			\RCitem{PC9}{\DONE{El paso 5 de la TP podría ser general como ``Verifica que los datos ingresados sean correctos según la regla de negocio...'' para que se verifique la contraseña, en todos los campos se verifica la longitud}}{05/11}
% 			\RCitem{PC10}{\DONE{Hace falta mencionar el mensaje de longitud}}{05/11}
% 			\RCitem{PC11}{\DONE{Falta mencionar el mensaje 22 en la sección de errores}}{05/11}
% 			\RCitem{PC12}{\DONE{Falta el mensaje de que se ha excedido la longitud}}{05/11}
% 			\RCitem{PC13}{\DONE{Falta el mensaje de formato incorrecto}}{05/11}
% 			\RCitem{PC14}{\DONE{En la trayectoria alternativa C el campo no debería de indicarse}}{05/11}
% 			\RCitem{PC15}{\DONE{En el paso 8 indicar que se trata de una pantalla en lugar de decir ``figura''}}{05/11}
% 			\RCitem{PC16}{\DONE{¿Se hablará de cómo cerrar la sesión?, en el sistema pasado se documento ``cerrar sesón'' como parte de la trayectoria principal del caso de uso ``Ingresar al sistema'' }}{05/11}
% 			\RCitem{PC17}{\DONE{En el segunto punto de extensión dice ``insripción''}}{05/11}
% 			
% 		\end{UClist}		
% 	}
%--------------------------------------------------------
	\UCccsection{Revisión Versión 0.3} % Anote la versión que se revisó.
	% FECHA: Anote la fecha en que se terminó la revisión.
	\UCccitem{Fecha}{11-11-14} 
	% EVALUADOR: Coloque el nombre completo de quien realizó la revisión.
	\UCccitem{Evaluador}{Natalia Giselle Hernández Sánchez}
	% RESULTADO: Coloque la palabra que mas se apegue al tipo de acción que el analista debe realizar.
	\UCccitem{Resultado}{Corregir}
	% OBSERVACIONES: Liste los cambios que debe realizar el Analista.
	\UCccitem{Observaciones}{
		\begin{UClist}
			% PC: Petición de Cambio, describa el trabajo a realizar, si es posible indique la causa de la PC. Opcionalmente especifique la fecha en que considera razonable que se deba terminar la PC. No olvide que la numeración no se debe reiniciar en una segunda o tercera revisión.
			\RCitem{PC1}{\DONE{Agregar a precondiciones el estado de la cuenta}}{Fecha de entrega}
			\RCitem{PC2}{\DONE{Agregar el paso de la trayectoria de validación del estado de la cuenta}}{Fecha de entrega}
			\RCitem{PC3}{\DONE{Agregar el mensaje de cuenta no activada a la sección de errores}}{Fecha de entrega}
			\RCitem{PC4}{\DONE{Verificar las ligas a los estados}}{Fecha de entrega}
			
		\end{UClist}		
	}
%--------------------------------------------------------

	\UCsection{Atributos}
	\UCitem{Actor}{\textbf{Auxiliar}}
	\UCitem{Propósito}{Consultar tratamientos de pacientes.}
	\UCitem{Entradas}{
        Ninguna.
    }
	\UCitem{Salidas}{
		\begin{UClist}
			\UCli Nombre de paciente.
			\UCli Nombre de tratameinto.
			\UCli Fecha en que se expidió el tratamiento.
			\UCli Estado del tratamiento.
			\UCli Número de medicamentos del tratamiento.
		\end{UClist}
	}
	\UCitem{Precondiciones}{
		{\bf Interna:} El actor debe estár registrado en el sistema.
		{\bf Interna:} El actor debe tener al menos un paciente con un tratamiento agregado.
		}
	\UCitem{Postcondiciones}{
	    {\bf Interna:} El actor podrá consultar la información completa del tratamiento.	
	}
    \UCitem{Reglas de negocio}{
    	Ninguna.
    }
	\UCitem{Mensajes}{
		Ninguno.
	}
	\UCitem{Tipo}{Primario.}
	\UCitem{Fuente}{
	}
 \end{UseCase}

 \begin{UCtrayectoria}
 	
 	\UCpaso [\UCactor] Da clic sobre el ícono \textbf{Consultar Pacientes} en la pantalla \textbf{Menú Principal}.
 	
 	\UCpaso Muestra la pantalla	\textbf{Gestión Paciente}.
% 	\cdtIdRef{IUP7}{Consultar Tratamiento}. 
 	
 	\UCpaso [\UCactor] Da clic sobre el paciente del cuál quieres consultar la información de su tratamiento en la pantalla \textbf{Gestión Paciente}.
 	
 	\UCpaso Obtiene el nombre del paciente.
 	
 	\UCpaso Obtiene el nombre del tratamiento.
 	
 	\UCpaso Obtiene la fecha de expedición del tratamiento.
 	
 	\UCpaso Obtiene el estado del tratamiento.
 	
 	\UCpaso Obtiene el número de medicamentos.
 	
 	\UCpaso Muestra la pantalla \textbf{Información del Paciente}
% 	\cdtIdRef{IUP7}{Consultar Tratamiento}.\refTray{A}
 	
 	\UCpaso Muestra primero los tratamientos activos, le siguen los incompletos, después los interrumpidos y al final los completados.\label{cud6:Acciones}
 	
    
 \end{UCtrayectoria}

 \begin{UCtrayectoriaA}{A}{El actor no cuenta con un paciente con un tratamiento registrado}
 	\UCpaso Muestra el mensaje \textbf{MSG No hay pacientes con tratamientos} en la pantalla \textbf{Gestión Pacientes}.
% 	\cdtIdRef{MSG9}{No hay tratamientos} en la pantalla \cdtIdRef{IUP7}{Consultar Tratamiento}
 	
 	\UCpaso [\UCactor] Presiona el botón \cdtButton{Agregar Nuevo Tratamiento} del mensaje. \refTray{B}

   	\UCpaso Extiende al caso de uso \textbf{CUA8: Agregar Tratamiento a Paciente}.
 \end{UCtrayectoriaA}
 

 
 
\subsection{Puntos de extensión}


\UCExtensionPoint
{El actor requiere editar un tratamiento}
{ Paso \ref{cud6:Acciones} de la trayectoria principal}
{\cdtIdRef{CUD8 }{Editar Tratamiento de Pacientes}}

 

