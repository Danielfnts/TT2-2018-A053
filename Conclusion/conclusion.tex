\newpage
\section{Conclusión}

Una vez realizadas las actividades definidas en los sprints de la metodología adoptada y analizados los resultados de las pruebas podemos concluir que se ha cumplido con el análisis y desarrollo del Asistente Móvil para el Seguimiento de Tratamientos Médicos (Rem-Pills) y que tanto el objetivo general, así como los objetivos específicos, se han cumplido de manera exitosa.\\

Se llego a esta conclusión debido a su aceptación que tuvo con los diferentes usuarios que fueron considerados para realizar las pruebas del sistema. Además de que nos hicieron muchos comentarios y observaciones para mejorar el diseño y las características con las que nuestro asistente móvil se podría enriquecer las cuales fueron consideradas para la versión final.\\

Durante el desarrollo de la aplicación movil (Rem-Pills) nos vimos impulsados a aprender a desarrollar en una tecnología nueva para nosotros, que a pesar de que cuenta con una gran ventaja que es que el 90\% del código se pueda compartir entre las plataformas de Android y de iOS aun existe el 10\% en el que tuvimos que aprender nociones básicas de desarrollo para iOS y Android. Por lo cual, para lograr los objetivos especificados en este trabajo terminal, nos dimos a la tarea de investigar e implementar información acerca del funcionamiento de Xamarin Forms, Xamarin Android y Xamarin iOS.\\

%Durante el desarrollo del asistente móvil para el seguimiento de tratamientos médicos (Rem-Pills) tuvimos algunas dificultades debido a que existe muy poca documentación sobre el desarrollo en Xamarin y esa herramienta era desconocida para nosotros. Por lo cual, para lograr los objetivos especificados en este trabajo terminal, nos dimos a la tarea de investigar e implementar la poca información encontrada.\\

La metodología adoptada y utilizada para este trabajo terminal nos permitió realizar los cambios necesarios de acuerdo a las necesidades que se fueron presentando durante todo este proceso que conlleva el trabajo terminal, es importante mencionar que gracias a la forma en que está desarrollado el asistente móvil nos permite reutilizar los módulos de manera independiente para proyectos futuros o añadir más módulos para enriquecerlo.\\

En conclusión, el asistente móvil se pudo utilizar de manera satisfactoria por los usuarios, y con todo lo mencionado anteriormente, vemos la viabilidad de que el sistema Rem-Pills pueda ser utilizado por las personas para ayudar al seguimiento de los tratamientos médicos.





