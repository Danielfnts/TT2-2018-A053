\begin{UseCase}{CUP19}{Consultar Alertas}
    {
    	Un paciente debe de estár conciente de cuáles son los medicamentos que son de vital importancia consumir, es por eso que debe de contar con una visualización de estos.
    }
    \UCitem{Versión}{1.0}
    \UCccsection{Paciente}
    \UCitem{Autor}{Daniel Josue Fuentes Hernández}
    \UCccitem{Evaluador}{Victor Arquimedes Estrada Machuca}
    \UCitem{Operación}{Consultar alarmas}
    \UCccitem{Prioridad}{Baja}
    \UCccitem{Complejidad}{Baja}
    \UCccitem{Volatilidad}{Baja}
    \UCccitem{Madurez}{Alta}
    \UCitem{Estatus}{Terminado}
    \UCitem{Fecha del último estatus}{05 de Octubre del 2018}


%--------------------------------------------------------

	\UCsection{Atributos}
	\UCitem{Actor}{\textbf{Paciente}}
	\UCitem{Propósito}{Consultar alarmas.}
	\UCitem{Entradas}{
        Ninguna
    }
	\UCitem{Salidas}{\begin{UClist} 
			\UCli Nombre del medicamento.
			\UCli Tratamiento al que pertenece
			\UCli Contacto de emergencia a quien contactar
	\end{UClist}}
	\UCitem{Precondiciones}{
		{\bf Interna:} Que exista un medicamento con clasificación \textbf{Importante}.
	}
	\UCitem{Postcondiciones}{
	    Ninguna.	
	}
    \UCitem{Reglas de negocio}{

            Ninguna.
    }
	\UCitem{Mensajes}{
	    \begin{UClist}
	    \UCli \cdtIdRef{MSG1}{Falta un dato requerido para efectuar la operación solicitada}: Se muestra en la pantalla \cdtIdRef{IUP4}{Datos Personales} cuando el actor omitió un dato marcado como requerido.
		\UCli \cdtIdRef{MSG3}{Correo electrónico y/o contraseña incorrecto}: Se muestra en la pantalla \cdtIdRef{IUP4}{Datos Personales} indicando que el correo electrónico y/o contraseña son incorrectos.
		\UCli \cdtIdRef {MSG24}{Validación de contraseña incorrecta}: Se muestra en la pantalla \cdtIdRef{IUP4}{Datos Personales} indicando que la confirmación de contraseña no coincide con la contraseña ingresada.
		
	    \end{UClist}
	}
	\UCitem{Tipo}{Primario.}
	\UCitem{Fuente}{
	}
 \end{UseCase}

 \begin{UCtrayectoria}
 	
 	
 	\UCpaso [\UCactor] Presiona el ícono \textbf{Consultar Alarmas} en la pantalla \textbf{Gestión recordatorios}
 	\UCpaso Verifica que haya alarmas. \refTray{A}
 	\UCpaso Obtiene el nombre del medicamento.
 	\UCpaso Obtiene el tratamiento al que pertenece.
 	\UCpaso Obtiene el contacto de emergencia que tiene que contactar.
 	\UCpaso Muestra la pantalla \cdtIdRef{IUP19}{Consultar Alertas}
     
 \end{UCtrayectoria}

 \begin{UCtrayectoriaA}{A}{El actor no tiene alarmas.}
	\UCpaso Muestra el mensaje \cdtIdRef{MSG26}{No hay alertas}.
	\UCpaso Muestra la pantalla \textbf{Gestión de Recordatorios}.
	
\end{UCtrayectoriaA}

